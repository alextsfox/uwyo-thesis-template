\chapter{Examining Examinations}
\section{Introduction}
\lipsum[1]

\section{Methods}
\lipsum[2]

\section{Results}
\lipsum[3]
\lipsum[4]\cite{bosch_review_1982}\lipsum[5]

\section{Discussion}
\lipsum[5]
\begin{figure}
    \centering
    \includegraphics[width=1\linewidth]{figs/budyko.png}
    \caption[Budyko plot]{Budyko-type plot for all four sites. Each single point represents data from within one water year. The solid lines represents the expected Budyko-type relationship from \citeA{zhang_response_2001} and the dashed lines represent the physical limits on ET within a Budyko-type framework. The vertical dotted lines represent the boundary between water-limited and energy-limited conditions. All plots show evaporative fraction (ET/P) on the y-axis plotted against aridity index (PET/P) on the x-axis. For (a), x- and y-values were computed using data from the entire water year. For (b), (c), and (d), x- and y-values were computed using data from only one season within each water year; winter includes Oct-March, spring includes Apr-Jun, and summer includes Jul-Sep.}
    \label{fig:budyko_curve}
\end{figure}
\lipsum[6]

\section{Conclusion}
\lipsum[7]